\documentclass{marckit_cv}

\setname{Marc-André}{Labelle}
\setaddress{Montreal/Quebec}
\setmobile{514-568-6836}
\setmail{marckitmail@gmail.com}
\setlinkedinaccount{https://www.linkedin.com/in/marckit/}
\setwebsite{https://marckit.com/}
\setgithubaccount{https://github.com/marckit}
\setgitlabaccount{https://gitlab.com/marckit}
\setthemecolor{blue} %color

\begin{document}
  %Create header
  \headerview
  \vspace{1ex}
  %Sections
  %
  %Education
  \section{Éducation}
  \datedexperience{UQAM (Université du Québec À Montréal)}
                  {Obtenu en Septembre 2021}
  \explanation{Baccalauréat en informatique et génie logiciel}
  \explanationdetail{\coloredbullet\
    \href{https://etudier.uqam.ca/programme?code=7416}
         {Il s’agit d’études informatiques dont les langages étudiés
                                            sont principalement Java, C et C++}
  }
  %
  \datedexperience{Université Laval}{Obtenu en Mai 2019}
  \explanation{Baccalauréat en génie alimentaire}
  \explanationdetail{\coloredbullet\
    \href{https://www.ulaval.ca/les-etudes/programmes/repertoire/details/baccalaureat-en-genie-alimentaire-b-ing.html}
    {Combinaison de notions de génie chimique et de scienece
                          \& technologie des aliments en génie alimentaire}\par
  }
  % Experience
  \section{Expérience}
  %
  \datedexperience{Entologik}{Août 2020 - Avril 2021}
  \explanation{Responsable technologique et alimentaire}
  \explanationdetail{\coloredbullet\
    \href{http://www.entologik.com}
         {Croissance de grillons, petit programme python
                                           et maintenance des ordinateurs}\par}
  %
  \datedexperience{Happy Yak}{Mai 2019 - Juillet 2019}
  \explanation{\href{http://www.happyyak.ca/}
                    {Stagiaire en production alimentaire}}
  %
  \datedexperience{Conagra Brands}{August 2015 - November 2015}
  \explanation{\href{http://www.conagrabrands.ca/fr/marques}{Sanitation}}
  %
  % Skills
  \section{Compétences}
  %
  \newcommand{\skillone}{\createskill{Langages de programmation}
    {\textbf{\emph{Expérimenté:}} Python \cpshalf Java \cpshalf
     \textbf{\emph{Familié:}} Javascript \cpshalf Bash
      \cpshalf Java \cpshalf C \cpshalf C++ \cpshalf MATLAB
    }
  }
  %
  \newcommand{\skilltwo}{\createskill{Outils de développement}
                                 {GIT \cpshalf CLI \cpshalf Méthodologie Agile
                                                 \cpshalf Cylce de vie DevOps}}
  %
  \newcommand{\skillthree}{\createskill{Déploiement d'interface}
                                        {React \cpshalf Angular \cpshalf Kivy}}
  %
  \newcommand{\skillfour}{\createskill{Interface utilisateur}
                    {Tailwind \cpshalf Bootstrap \cpshalf Angular Material UI}}
  %
  \newcommand{\skillfive}{\createskill{Langues}
    {\textbf{\emph{Maternelle:}} Français \cpshalf
     \textbf{\emph{Courante:}} Anglais \cpshalf
     \textbf{\emph{Débutant:}} Espagnol}}
  %
  \createskills{\skillone, \skilltwo, \skillthree, \skillfour, \skillfive}
  %
  % Other Experiences
  \section{Autres expériences}
  \datedexperience{\href{https://www.instagram.com/parcelles.austin/}
                                                          {Parcelles}}
                  {Agriculture}
  \datedexperience{\href{https://www.tableedeschefs.org/}
                                       {La tablée des chefs}}
                  {Cuisinier}
  \datedexperience{\href{https://www.lanticrogers.com/}
                                         {Sucre Lantic}}
                  {Vérification de l'application des normes ISO et HACCP}
  \datedexperience{\href{https://www.dominiongrimm.ca/fr/home.html}
                                                {Dominion et Grimm}}
                  {Testes d'équipements}

  % Experience
  \section{Passe-temps}
  \newcommand{\extraone}{%
    \emph{Musique}: Joue guitar, piano, basse
  }
  \newcommand{\extratwo}{%
    \emph{Sports}: Hockey, Tennis, Yoga, ect.
  }

  \newcommand{\listofextras}{\extraone, \extratwo}

  \createbullets{\listofextras}

  %Footnote
  \createfootnote
\end{document}
